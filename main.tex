\documentclass[]{report}

\usepackage{etoolbox}
\makeatletter
\patchcmd{\thebibliography}{%
	\chapter*{\bibname}\@mkboth{\MakeUppercase\bibname}{\MakeUppercase\bibname}}{%
	\section*{References}}{}{}
\makeatother

% Title Page
\title{Report of Activities, 2016}
\author{Jo\~{a}o Elias Brasil Bentes J\'{u}nior}

\begin{document}
\maketitle

\section*{Research Description} \label{research-description}

This research aims at: (i) designing a general architecture for the continuous delivery of services at home and elsewhere; and (ii) proposing a framework which aid in the development of continuous services. The research questions are:
\begin{itemize}
	\item What is the minimum infrastructure to deliver the same service indoors and outdoors?
	\item What are the functional and non-functional requirements for the architecture?
	\item To what extent can the same level of service personalization be kept indoors and outdoors?
\end{itemize}
Keywords: user mobility, service continuity, service provision, smart home, smart city.

\section*{Research Progress} 

\begin{itemize}
	\item Literature studies
	\begin{itemize}
		\item Smart Home Systems and Ambient Assisted Living Frameworks. Most relevant papers: \cite{acampora2013survey,blackman2016ambient,memon2014ambient,alam2012review}
		\item Context-Aware Systems. Most relevant papers: \cite{henricksen2005middleware,perera2014context}
		\item Smart City Architectures and Sensing as a Service paradigm. Most relevant papers: \cite{perera2014sensing,perera2014sensor,sheng2013sensing,zaslavsky2013sensing}
		\item Middlewares for IoT and code migration. More relevant papers: \cite{razzaque2016middleware,yu2013application,raychoudhury2013middleware}
		\item Standards for Health Care Systems proposed by Continua Alliance. Most relevant documents: \cite{schreier2014internet,rickardsson2016patient,clarke2007developing}
	\end{itemize}
	\item Study of closely related PhD thesis:
	\begin{itemize}
		\item \textit{Abstract service model for adaptive provision in ambient assistive living},  Mahmoud Ghorbel \cite{ghorbel2008abstract}.
		\item \textit{Architectural styles and the design of network-based software architectures}, Roy T. Fielding \cite{fielding2000architectural}.
		\item \textit{KSPOT: a network-aware framework for energy-efficient data acquisition in wireless sensor networks}, Panayiotis G. Andreou \cite{andreou2011kspot}.
		\item \textit{Architecting Smart Home Environments for Healthcare: A Database-Centric Approach}, Wagner Ourique de Morais \cite{ourique2015architecting}. 
	\end{itemize}
	\item Design of the application scenario: Ubiquitous fall detection system.
	\item State of the art survey towards the Ubiquitous fall detection scenario [Ongoing work].	
\end{itemize}

\section*{Publications}

Two conference papers were written and published during my first year. Both in collaboration with other research teams. The paper 1) was the result of collaboration with IPAL (Image and Pervasive Lab), Singapore. The paper 2) is a result of an internal collaboration with Intelligent Lab of Halsmtad University.

\begin{enumerate}
	\item Aloulou H., Abdulrazak B., Endelin R., \textbf{Bentes J.}, Tiberghien T., Bellmunt J.. Simplifying Installation and Maintenance of Ambient Intelligent Solutions Toward Large Scale Deployment. In International Conference on Smart Homes and Health Telematics 2016 May 25 (pp. 121-132). Springer International Publishing.
	\item Lundstr\"{o}m, J., de Morais, W.O., Menezes, M., Gabrielli., C, \textbf{Bentes, J.}, SantaAnna, A., Synnot, J., Nugget, C.. Halmstad Intelligent Home - Capabilities and Opportunities. In The 3rd EAI International Conference on IoT Technologies for HealthCare 2016 October 17. Springer International Publishing. [Accepted for publication]
\end{enumerate}

\noindent Also, a paper raised from my master thesis was published in 2016:

\begin{itemize}
	\item \textbf{Bentes, J.}, Trevisan, D. and Viterbo, J., 2016, January. Expanding the Coverage of Ambient Assisted Living Systems. In Proceedings of the 2016 49th Hawaii International Conference on System Sciences (HICSS) (pp. 3268-3277). IEEE Computer Society. 
\end{itemize}

\section*{Courses}

\begin{itemize} 
	\item Completed courses.
	\begin{itemize}
		\item Introductory Course for PhD Students, 7.5 credits.
	\end{itemize}	
	\item Ongoing courses (finishing in 2016).
		\begin{itemize}
			\item Advanced Python Programming, 1.5 credits.
			\item Distributed Real Time Systems, 7.5 credits.
			\item The Quantitative Research Process - from idea to contribution, 7.5 credits.
		\end{itemize}
\end{itemize}

\section*{Events}

\begin{itemize} 
	\item Summer schools
		\begin{itemize}
			\item August: \textit{Health Innovation Summer School}, Halmstad, Sweden.
		\end{itemize}
	\item Scientific conferences
		\begin{itemize}
			\item June: \textit{International Conference on Biometrics (ICB 2016)}, Halmstad, Sweden.
			\item September: \textit{The International Symposium on Wearable Computers (ISWC 2016)}, Heidelberg, Germany.
			\item September: \textit{ACM International Joint Conference on Pervasive and Ubiquitous Computing (UbiComp 2016)}, Heidelberg, Germany.
		\end{itemize}
	\item Workshops
		\begin{itemize}
			\item February: \textit{ITE (Information Science, Computer and Electrical Engineering) PhD Conference 2016}, February, \"{A}ngelholm, Sweden.
			\item June: \textit{CAISR (Center for Applied Intelligent Systems Research) Industrial Workshop}, Halmstad, Sweden.
			\item September: \textit{Workshop on contactless assessment of vital signs}, Halmstad, Sweden.
		\end{itemize}
	\item Given Seminars
		\begin{itemize}
			\item May: \textit{Handover in Ambient Assisted Living}, Halmstad University, Sweden.
		\end{itemize}
	\item Others
		\begin{itemize}
			\item September: \textit{3rd East Sweden Hackathon}, Link\"{o}ping, Sweden.
		\end{itemize}
\end{itemize}

\section*{Other relevant activities}
\begin{itemize}
	\item Collaboration attempts
		\begin{itemize}
			\item M\"{a}lardalen University: Part of the group that visited on May 19 to discuss potential collaboration with the research group of Biomedical Engineering, which is lead by Professor Maria Lind\'{e}n. 
			\item IPAL (Image and Pervasive Lab), Singapore: Skype meetings with the IPAL's research team to discuss potential collaboration between with the laboratory, which is lead by Mounir Mokhtari.
			\item UFF (Universidade Federal Fluminense): Skype meetings with Prof. Jos\'{e} Viterbo, who was my Master's supervisor, to discuss potential joint conference papers with Lab Tempo, which is lead by him.
		\end{itemize}
	\item Project discussions
	\begin{itemize}
		\item REMIND (The use of computational techniques to improve compliance to reminders within smart environments): contributed by writing a personal mobility plan for visiting partners (companies) outside Sweden for 2 months per year in the next 3 years.
	\end{itemize}
	\item Activities and Societies
	\begin{itemize}
		\item Health Tech Meeting: every second Thursday, discussions about research in health technology.
		\item Halmstad Research Student Society (HRSS): board meeting every second month. I have the role of Secretary.
	\end{itemize}
\end{itemize}

\section*{Other things that affect studies}

Due to the lost of one my father on June 16th, I went back to Brazil between June 16th and June 30th. This episode was properly informed to the \textit{SEBPE (Servi\c{c}o de Bolsas de P\'{o}s Gradua\c{c}\~{a}o e Pesquisa no Exterior)} on July 11th. 

\bibliography{references}{}
\bibliographystyle{plain}

\end{document}          
