\documentclass[]{report}

\usepackage{etoolbox}
\makeatletter
\patchcmd{\thebibliography}{%
	\chapter*{\bibname}\@mkboth{\MakeUppercase\bibname}{\MakeUppercase\bibname}}{%
	\section*{References}}{}{}
\makeatother

% Title Page
\title{Report of Activities, 2016}
\author{Jo\~{a}o Elias Brasil Bentes J\'{u}nior}

\begin{document}
\maketitle

\section*{Research Description} \label{research-description}

The clear benefits of care at home, particularly regarding the social integration and autonomy of elderly people, strongly motivate the development of mechanisms to improve the quality of life in such scenarios. Ambient Intelligence (AmI) systems that focus on providing health assistance at home for elderly people are called Ambient Assisted Living (AAL) systems~\cite{sun2009promises}. AAL Systems can be defined as a set of ubiquitous technologies, named AAL Services, that are embedded in a living space, named AAL Space, and aim at extending the time elderly and disabled people can live independently in some of their preferred places, such as their homes, neighborhoods, offices, gyms, health facilities etc~\cite{living2011ambient}.

Generally, their own house is the place where elderly people prefer to stay longer~\cite{aloulou2013deployment}~\cite{sixsmith2014healthy}. Hence, AAL services for assisting and monitoring home dwellers are not commonly tailored to extrapolate the boundaries of the house, being generally provided only inside well-delimited areas. However, some AAL systems might take advantage of some level of continuity. When users leave their houses, they could benefit from some services commonly provided by AAL environments, if such services were also made available beyond the house's boundaries. In other words, the migration of the users through different environments, i.e., the user's mobility, imposes a demand for the expansion of the boundaries of AAL coverage, which may be achieved by the integration of several different AAL Spaces or by the empowerment of outdoor services~\cite{Cook2009_2} with the support of mobile devices~\cite{Persona}.

This research aims at: (i) designing a general architecture for the continuous delivery of services at home and elsewhere; and (ii) proposing a framework which aid in the development of continuous services. The research questions are:
\begin{itemize}
	\item What is the minimum infrastructure to deliver the same service indoors and outdoors?
	\item What are the functional and non-functional requirements for the architecture?
	\item To what extent can the same level of service personalization be kept indoors and outdoors?
\end{itemize}

Keywords: user mobility, service continuity, service provision, smart home, smart city.

\section*{Activities} 

\begin{itemize}
	\item Literature studies
	\begin{itemize}
		\item Smart Home Systems and Ambient Assisted Living Frameworks. Most relevant papers: \cite{acampora2013survey,blackman2016ambient,memon2014ambient,alam2012review}
		\item Context-Aware Systems. Most relevant papers: \cite{henricksen2005middleware,perera2014context}
		\item Smart City Architectures and Sensing as a Service paradigm. Most relevant papers: \cite{perera2014sensing,perera2014sensor,sheng2013sensing,zaslavsky2013sensing}
		\item Middlewares for IoT and code migration. More relevant papers: \cite{razzaque2016middleware,yu2013application,raychoudhury2013middleware}
		\item Standards for Health Care Systems proposed by Continua Alliance. Most relevant documents: \cite{schreier2014internet,rickardsson2016patient,clarke2007developing}
	\end{itemize}
	\item Study of closely related PhD thesis:
	\begin{itemize}
		\item \textit{Abstract service model for adaptive provision in ambient assistive living},  Mahmoud Ghorbel \cite{ghorbel2008abstract}.
		\item \textit{Architectural styles and the design of network-based software architectures}, Roy T. Fielding \cite{fielding2000architectural}.
		\item \textit{KSPOT: a network-aware framework for energy-efficient data acquisition in wireless sensor networks}, Panayiotis G. Andreou \cite{andreou2011kspot}.
		\item \textit{Architecting Smart Home Environments for Healthcare: A Database-Centric Approach}, Wagner Ourique de Morais \cite{ourique2015architecting}. 
	\end{itemize}
	\item Write of internal report describing my application scenario: ubiquitous activity/inactivity detection system.
	\item Networking
	\begin{itemize}
		\item M\"{a}lardalen University: Part of the group that visited on May 19 to discuss potential collaboration with the research group of Biomedical Engineering, which is lead by Professor Maria Lind\'{e}n. 
		\item IPAL (Image and Pervasive Lab), Singapore: Mediated Skype meetings with the IPAL's research team to discuss potential collaboration between with the laboratory, which is lead by Mounir Mokhtari.
		\item UFF (Universidade Federal Fluminense): Mediated Skype meetings with Prof. Jos\'{e} Viterbo, who was my Master's supervisor, to discuss potential joint conference papers with Lab Tempo, which is lead by him.
	\end{itemize}
	\item Contributions in other projects
	\begin{itemize}
		\item REMIND (an European Union granted project): the aim of this project is to create an International and Intersectoral network to facilitate the exchange of staff to progress developments in reminding technologies for persons with dementia which can be deployed in smart environments. My contribution on this project was writing my personal mobility plan to visit some industrial partners for one or two months per year in the next three years. Those partners are in Italy, Spain, Northern Ireland and Norway.
	\end{itemize}
	\item Groups and societies
	\begin{itemize}
		\item Health Tech Meeting: every second Thursday, discussions about research in health technology.
		\item Halmstad Research Student Society (HRSS): board meeting every second month. I have the role of Secretary.
	\end{itemize}
\end{itemize}

\section*{Publications}

The following papers are currently in the writing process and should be submitted for conferences this year.

\begin{itemize}
	\item Conference full-paper: state of the art survey towards the application scenario (ubiquitous activity/inactivity detection system).
	\item Conference short-paper: overview of my research goals for the Doctoral Symposium in ACM/IFIP/USENIX Middleware 2016.
\end{itemize}

\noindent Two conference full-papers were written and published during my first year. Both in collaboration with other research teams. The paper 1) was the result of collaboration with IPAL (Image and Pervasive Lab), Singapore. The paper 2) is a result of an internal collaboration with Intelligent Lab of Halsmtad University.

\begin{enumerate}
	\item Aloulou H., Abdulrazak B., Endelin R., \textbf{Bentes J.}, Tiberghien T., Bellmunt J.. Simplifying Installation and Maintenance of Ambient Intelligent Solutions Toward Large Scale Deployment. In International Conference on Smart Homes and Health Telematics 2016 May 25 (pp. 121-132). Springer International Publishing.
	\item Lundstr\"{o}m, J., de Morais, W.O., Menezes, M., Gabrielli., C, \textbf{Bentes, J.}, SantaAnna, A., Synnot, J., Nugget, C.. Halmstad Intelligent Home - Capabilities and Opportunities. In The 3rd EAI International Conference on IoT Technologies for HealthCare 2016 October 17. Springer International Publishing. [Accepted for publication]
\end{enumerate}

\noindent Also, a paper raised from my master thesis was published in 2016:

\begin{itemize}
	\item \textbf{Bentes, J.}, Trevisan, D. and Viterbo, J., 2016, January. Expanding the Coverage of Ambient Assisted Living Systems. In Proceedings of the 2016 49th Hawaii International Conference on System Sciences (HICSS) (pp. 3268-3277). IEEE Computer Society. 
\end{itemize}

\section*{Courses}

Education on doctoral level is ended with doctoral degree or licentiate degree. The doctoral student also has the possibility to get a licentiate degree as a partial stage in the education. For doctoral degree the following is required:(i) approved courses of at least 60 credits; and (ii) approved scientific thesis of at least 150 credits. Thesis and courses shall together be at least 240 credits.

\begin{itemize} 
	\item Completed courses.
	\begin{itemize}
		\item Introductory Course for PhD Students, 7.5 credits.
	\end{itemize}	
	\item Ongoing courses (finishing in 2016).
		\begin{itemize}
			\item Advanced Python Programming, 1.5 credits.
			\item Distributed Real Time Systems, 7.5 credits.
			\item The Quantitative Research Process - from idea to contribution, 7.5 credits.
		\end{itemize}
\end{itemize}

\section*{Events}

\begin{itemize} 
	\item Summer schools
		\begin{itemize}
			\item August: \textit{Health Innovation Summer School}, Halmstad, Sweden.
		\end{itemize}
	\item Scientific conferences
		\begin{itemize}
			\item June: \textit{International Conference on Biometrics (ICB 2016)}, Halmstad, Sweden.
			\item September: \textit{The International Symposium on Wearable Computers (ISWC 2016)}, Heidelberg, Germany.
			\item September: \textit{ACM International Joint Conference on Pervasive and Ubiquitous Computing (UbiComp 2016)}, Heidelberg, Germany.
		\end{itemize}
	\item Workshops
		\begin{itemize}
			\item February: \textit{ITE (Information Science, Computer and Electrical Engineering) PhD Conference 2016}, February, \"{A}ngelholm, Sweden.
			\item June: \textit{CAISR (Center for Applied Intelligent Systems Research) Industrial Workshop}, Halmstad, Sweden.
			\item September: \textit{Workshop on contactless assessment of vital signs}, Halmstad, Sweden.
		\end{itemize}
	\item Given Seminars
		\begin{itemize}
			\item May: \textit{Handover in Ambient Assisted Living}, Intelligent Systems Lab, Halmstad University, Sweden.
			\item October [Planned]: \textit{Challenges for continuous provisioning of assisted living services in Smart Cities}, Intelligent Systems Lab, Halmstad University, Sweden.
			\item November [Planned]: \textit{Challenges for continuous provisioning of assisted living services in Smart Cities}, Computing and Communication Lab, Halmstad University, Sweden.
		\end{itemize}
	\item Others
		\begin{itemize}
			\item September: \textit{3rd East Sweden Hackathon}, Link\"{o}ping, Sweden.
		\end{itemize}
\end{itemize}

\section*{Other things that affect studies}

Due to the lost of my father on June 16th, I went back to Brazil between June 16th and June 30th. This episode was properly informed to the \textit{SEBPE (Servi\c{c}o de Bolsas de P\'{o}s Gradua\c{c}\~{a}o e Pesquisa no Exterior)} on July 11th. 

\bibliography{references}{}
\bibliographystyle{plain}

\end{document}          
